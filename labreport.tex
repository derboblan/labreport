\documentclass[a4paper, ngerman, 12pt]{scrartcl}
\usepackage{amssymb}
\usepackage{ifxetex}
\ifxetex
    % XeLaTeX
    \usepackage{fontspec,xunicode,xltxtra}
    \usepackage{polyglossia}
    \setmainlanguage[spelling=new,babelshorthands=true]{german}
    \setsansfont{Gillius ADF}
    \usepackage{ebgaramond}
    \usepackage[]{unicode-math}
    %    \setmainfont{XITS}
    %    \setmathfont{XITS Math}
\else
    % default: pdfLaTeX
    \usepackage[utf8]{inputenc}
    \usepackage[ngermanb]{babel}
    \usepackage[T1]{fontenc}
    \usepackage[babel=true]{microtype}
    \usepackage{lmodern}
\fi
\usepackage{amsmath}

\usepackage{eurosym}
\usepackage{parskip}
\usepackage{svg}
\usepackage{lscape}
\usepackage[inner=3.5cm,outer=3.5cm,top=3.3cm,bottom=6.6cm]{geometry}

\usepackage[nottoc]{tocbibind}
\usepackage[fixlanguage]{babelbib}
%\selectbiblanguage{german}

\usepackage{graphicx}
\usepackage{caption}
\usepackage{xcolor}
    \definecolor{lblue}{cmyk}{0.92, 0.7, 0, 0}            % #2020c0
\usepackage[hyphens]{url}
\usepackage[bookmarks]{hyperref}
    \hypersetup{
        colorlinks=true,  % false: boxed links; true: colored links
        linkcolor=lblue,   % color of internal links
        citecolor=lblue,   % color of links to bibliography
        filecolor=lblue,   % color of file links
        urlcolor=lblue     % color of external links
    }

\usepackage{fancyhdr}
\setlength{\headheight}{15pt}
\pagestyle{fancy}
\fancyhf{}
\lhead[\nouppercase\leftmark]{}
\rhead[]{\nouppercase\rightmark}
\lfoot[\thepage]{}
\rfoot[]{\thepage}

%\newfontfamily\sectionfont[Color=MSLightBlue]{Times New Roman}

\linespread{1.2} % Zeilenhöhe wird auf Faktor 1.25 gesetzt
\setlength{\parindent}{0cm}
\graphicspath{{img/}}
%\renewcommand*{\figureformat}{}
% Die Boxen zum zählen der Verdrängungsindikatoren als Matheoperatoren, damit das spacing einheitlich ist
\DeclareMathOperator{\bsq}{\blacksquare}
\DeclareMathOperator{\wsq}{\square}
\usepackage{pifont}
%\renewcommand{\labelitemi}{\ding{112}} % \blacktriangleright

\title{Hardware Security}
\subtitle{Labreport}


\date{}
\begin{document}
    \pagenumbering{roman}
    \begin{titlepage}
    \newgeometry{inner=3.45cm,outer=3.45cm,top=3.3cm,bottom=4.6cm}
        \begin{figure*}
            \begin{flushright}
%                \includegraphics[width=0.4\textwidth]{img/Logokurz.pdf}
            \end{flushright}
        \end{figure*}
        \vspace*{-1cm}
        \hrule
        \vspace{.2cm}
        \begin{center}
            \vfill
            \textsf{\textbf{\Huge{Hardware Security}}}\\[1.5em]
            \textsf{\LARGE{Labreport}}
            \vfill
        \end{center}
        \vfill
        \hrule
        \vspace{.2cm}
        \large{Berlin, October 2016}
    \end{titlepage}
    
    \pagenumbering{arabic}
    \setcounter{page}{1}
    
    \section*{Assignment 1}
% Installing VA testbacnhes.

\begin{itemize}
    \item Installing \& get acquainted with the lab environment (virtual machine)
    \item differences between combinatorial and sequential logic and how to implement in Verilog guided by some examples (halfadder, counter)
    \item simulation and testing in ISE
\end{itemize}

%\emph{Aufgabe: }Nach eingien einführenden Aufgabe, sollten wir einen UART Transmitter, dann einen UART-Reciver, und abschließend um beide miteinander zu Kombinieren Ein Echo auf dem FPGI zu Implementieren.

After some small introducing assignments we were asked to implement an UART-transmitter and an UART-reciver and test them via test bench. Afterwards we should combine these two to implement an echo on the FPGI.


%\emph{Bearbeitung und Lösung: }Der Empfänger hat 3 Zustände $z_0$, $z_1$, $z_2$.
The UART-reciver has three different states $z_0$, $z_1$, $z_2$.
\begin{itemize}


\item[$z_0$:] Initial state: Waiting for falling/rising edge, which is showing the begin of an UART communication.
%Ausgangszustand warten auf fallende/steigende Flanke, die den Beginn der seriellen UART Kommunikation anzeigt.
\item[$z_1$:] Receiving state: For every bit you have to wait the appropriate time according to the transfer rate, until you can read the next bit. The individual bits read will be saved to a byte array. After saving a bit yo have to shift all the bit in the array, so the next read Bit can be saved to the 1st position.
%Empfangszustand: Für jedes Bit muss entsprechend der Übertragungsrate gewartet werden, bis das nächste gelesen werden kann. Die eingelesenen einzel Bits werden in einem Byte Array gespeichert, dass nach speichern eines Wertes ein Bit geshiftet wird, um das nächste Bit zu speichern. 
\item[$z_2$:] Stop state: In this state the bit-string will be reset to its initial state. Furthermore, the valid bit will be set, to show that the saved byte can be read through the output.
%im Stoppzustand wird der Bit-String wieder in den Augangszustand versetzt, außerdem wird Valid-Bit gestezt. Das heißt, dass das Byte nun ausgelesen werden kann (am Output anliegt). 
\end{itemize}

%\emph{Bearbeitung und Lösung: }Der Sender hat 4 Zustände $z_0$, $z_1$, $z_2$, $z_3$.
The UART-transmitter got four states $z_0$, $z_1$, $z_2$, $z_3$.
\begin{itemize}


\item[$z_0$:] Initial state: In the initial state, the transmitter is waiting for an enable. The Output is set to \verb+0+.
%Ausgangszustand wartet auf enable. Ausgabe ist auf 0.
\item[$z_1$:] Sending state: The output will be set to the value of the zeroth bit of an 8-bit string. Furthermore, all the bits of the 8-bit string will be shifted by one position.
%Die Augabe, wird dem 0. Bit eines 8-Bit Strings gleichgesetzt und das 8-Bit Sting wird um eine Stelle verschoben.
\item[$z_2$:] Checking state: in this state will be checked if the byte where shifted 8 times. If not, the transmitter will go back in sending state. If yes, the transmitter will go into the finished state.
%Nachdem 8 Bits ausgelesen, bzw. gesendet wurden Wird der vorgang beendet.
\item[$z_3$:] Finished state: in this state the byte was fully transmitted and rdy will be set to 1 to show, that the transmitter is ready to send new data.
%rdy wird auf 1 gesetzt, um anzuzeigen, das er bereit ist neue Daten zu senden.  

\end{itemize}\newpage

\begin{figure}[h!]
    \begin{center}
%        \hrule\vspace{1em}
        \usetikzlibrary{arrows.meta}
\usetikzlibrary{calc,intersections,through,backgrounds}
\begin{tikzpicture}
% 	\tikzset{
% 	  every node/.style={scale=1.1}
% 	}	
	\tikzset{comp/.style={
		rectangle, draw=black, thick
	}}	
	\tikzset{component/.style={
		comp, minimum width=6cm, minimum height=6cm, very thick
	}}
	\tikzset{component_small/.style={
		comp, minimum width=2cm, minimum height=2cm, thick
	}}
	\tikzset{component_tiny/.style={
		comp, inner sep=0.1cm, semithick, right
	}}
	\tikzset{caption/.style={
		below right
	}}
	\tikzset{conn/.style={
		-{Latex[length=2mm]}
	}}
	
	% FPGA
	\node (FPGA) [component] at (0,0) {}
		% Caption
		node [caption] at (FPGA.south west) { \small{\textsf{\textbf{FPGA}}} }
		
		% In/-outputs links
		coordinate [yshift=4cm+0.4pt+0.666cm, label={ above right : \footnotesize{$\texttt{rx}_0$} }] (FPGA_rx0) at (FPGA.south west) % unten
		coordinate [yshift=4cm+0.4pt+1.333cm, label={ above right : \footnotesize{$\texttt{tx}_0$} }] (FPGA_tx0) at (FPGA.south west) % oben

		% In/outputs  rechts
		coordinate [yshift=4cm+0.4pt,                    label={ above left : \footnotesize{$\texttt{rx}_1$} }]      (FPGA_rx1)       at (FPGA.south east)  % unten
		coordinate [yshift=4cm+0.4pt+0.666cm, label={ above left : \footnotesize{$\texttt{tx}_1$} }]      (FPGA_tx1)       at (FPGA.south east) % mitte
		coordinate [yshift=4cm+0.4pt+1.333cm, label={ above left : \footnotesize{$\texttt{restart}$} }] (FPGA_restart) at (FPGA.south east) % oben
		
		% Internals
		node (Mux)          [component_tiny, shift={(0.25cm,-1.8cm) }]  at (FPGA.north west)   { \scriptsize{\textsf{\texttt{Mux}}} }
	;

	% Logic
	\node (Logic) at (FPGA.north) [comp, minimum height=1cm, minimum width=1.5cm, below right, shift={(0.6cm, -0.75cm)}] {}
		node [above right] at (Logic.south west) { \textsf{\footnotesize{\textbf{Logic}}} }
	;

	% Receiver
	\node (Receiver) at (FPGA.south east) [component_small, above left, shift={(-0.5, 0.75)}] {}
		% Caption
		node [caption] at (Receiver.south west) { \textsf{\footnotesize{\textbf{UART Recv.}}} }

		% Input rechts
		coordinate [yshift=1cm, label={ left : \scriptsize{\texttt{din}} }] (Receiver_din) at (Receiver.south east)

		% Outpus links
		coordinate [yshift=0.666cm,                 label={ right : \scriptsize{\texttt{valid}} }]           (Receiver_valid)           at (Receiver.south west) % unten
		coordinate [yshift=1.333cm+0.15cm, label={ right : \scriptsize{\texttt{data\_out}} }] (Receiver_data_out)    at (Receiver.south west) % oben
		coordinate [yshift=1.333cm-0.15cm,  label={ right : \scriptsize{$[7:0]$} }]                     (Receiver_data_out2) at (Receiver.south west) % mitte
	;

	% Receiver2
	\node (Receiver2) at (FPGA.south west) [component_small, above right, shift={(0.5, 0.75)}] {}
		% Caption
		node [caption] at (Receiver2.south west) { \textsf{\footnotesize{\textbf{UART Recv.}}} }

		% Input rechts
		coordinate [yshift=1cm, label={ right : \scriptsize{\texttt{din}} }] (Receiver2_din) at (Receiver2.south west)

		% Outpus links
		coordinate [yshift=0.666cm,                 label={ left : \scriptsize{\texttt{valid}} }]           (Receiver2_valid)           at (Receiver2.south east) % unten
		coordinate [yshift=1.333cm+0.15cm, label={ left : \scriptsize{\texttt{data\_out}} }] (Receiver2_data_out)    at (Receiver2.south east) % oben
		coordinate [yshift=1.333cm-0.15cm,  label={ left : \scriptsize{$[7:0]$} }]                     (Receiver2_data_out2) at (Receiver2.south east) % mitte
	;

	% Transmitter
	\node (Transmitter) at (FPGA.north west) [component_small, below right, shift={(1.35, -0.25)}] {}
		node [caption] at (Transmitter.south west) { \textsf{\footnotesize{\textbf{UART Trans.}}} }

		% Output links
		coordinate [yshift=1cm, label={ right: \scriptsize{\textsf{\texttt{dout}}} }] (Transmitter_dout) at (Transmitter.south west) % unten

		% Inputs links
		coordinate [yshift=0.666cm,                 label={ left : \scriptsize{\texttt{enable}} }]    (Transmitter_enable)   at (Transmitter.south east) % unten
		coordinate [yshift=1.333cm-0.15cm,  label={ left : \scriptsize{$[7:0]$} }]                  (Transmitter_data_in2)at (Transmitter.south east) % mitte
		coordinate [yshift=1.333cm+0.15cm, label={ left : \scriptsize{\texttt{data\_in}} }] (Transmitter_data_in)  at (Transmitter.south east) % oben	
	;

	% Computer
	\node (Computer) [component_small, below left, xshift=-1cm] at (FPGA.north west) {}
		% Caption
		node [caption] at (Computer.south west) { \small{\textsf{\textbf{Computer}}} }

		% In/outputs rechts
		coordinate [yshift=0.666cm, label={ left:\footnotesize{\texttt{tx}} }] (Computer_tx) at (Computer.south east) % unten
		coordinate [yshift=1.333cm, label={ left:\footnotesize{\texttt{rx}} }] (Computer_rx) at (Computer.south east) % oben
	;

	% Lock
	\node (Lock) [component_small, below right, xshift=1cm] at (FPGA.north east) {}
		% Caption
		node [caption] at (Lock.south west) { \small{\textsf{\textbf{Lock}}} }

		% In/outputs rechts
		coordinate [yshift=0.333cm, label={ right:\footnotesize{\texttt{tx}} }]   (Lock_tx)   at (Lock.south west) % unten
		coordinate [yshift=0.999cm, label={ right:\footnotesize{\texttt{rx}} }]   (Lock_rx)   at (Lock.south west) % mitte
		coordinate [yshift=1.666cm, label={ right:\footnotesize{\texttt{rst}} }] (Lock_rst)  at (Lock.south west) % oben
	;

	% Computer <-> FPGA
	\draw[conn]  (FPGA_tx0) -- (Computer_rx);
	\draw[conn] (Computer_tx) -- (FPGA_rx0);

	% FPGA <-> Lock
	\draw[conn] (FPGA_restart) -- (Lock_rst);
	\draw[conn] (FPGA_tx1) -- (Lock_rx);
	\draw[conn] (Lock_tx) -- (FPGA_rx1) ;
	
	% FPGA internal
		\draw[conn] (Logic.east) -| ([shift={(0.15cm, 0.566cm)}] Logic.east) -- (FPGA_restart);
		\draw[conn, name path=FPGA_rx0--FPGA_tx1] (FPGA_rx0) -| ([shift={(0.2cm, -1.375cm)}] FPGA_rx0) -- ([shift={(5.3cm, -1.375cm)}] FPGA_rx0) |- (FPGA_tx1); % Pass through Computer -> Lock
	
		% Connections to/from Receiver
		\draw[conn, name path=FPGA_rx1--Receiver_din] (FPGA_rx1) -- ([xshift=-0.2cm] FPGA_rx1) |- (Receiver_din); % rx1 -> Logic
		\draw[conn, very thick] (Receiver_data_out) -|  ([shift={(-0.2cm,0.6cm)}] Receiver_data_out) -| ([xshift=0.3cm] Logic.south);  % data_out -> Logic
		\draw[conn] (Receiver_valid) -| ([shift={(-0.4cm, 1.5cm)}] Receiver_valid) -| ([xshift=0.1cm] Logic.south); % valid -> Logic
		
		% Connections to/from Receiver2
		\draw[conn, name path=FPGA_rx0--Receiver2_din] (FPGA_rx0) -- ([xshift=0.2cm] FPGA_rx0) |- (Receiver2_din); % rx0 -> din
		\draw[conn, very thick] (Receiver2_data_out) -| ([shift={(0.2cm,0.9cm)}] Receiver2_data_out) -| ([xshift=-0.3cm] Logic.south); % data_out -> Logic 
		\draw[conn] (Receiver2_valid) -| ([shift={(0.4cm, 1.6cm)}] Receiver2_valid) -| ([xshift=-0.1cm] Logic.south); % valid -> Logic

		% Connections to/from Transmitter
		\draw[conn, very thick] ([yshift=-0.192cm] Logic.north west) -- ([yshift=-0.15cm] Transmitter_data_in); % Logic -> data_in
		\draw[conn] ([yshift=0.166cm] Logic.south west) -- (Transmitter_enable); % Logic -> enable
		\draw[conn] (Transmitter_dout) -| ([shift={(-0.15cm, -1.3cm)}] Transmitter_dout) -| ([xshift=0.1cm] Mux.south); % Transmitter -> Mux

		% Connections to/from Mux
		\draw[conn, name path=FPGA_rx1--Mux]  (FPGA_rx1) -- ([xshift=-0.5cm] FPGA_rx1) |-  ([shift={(-0.2cm, -1.55cm)}]  Transmitter_dout) -| ([xshift=-0.1cm] Mux.south); % rx1 -> Mux
		\draw[conn] ([xshift=-0.5cm] Logic.south) |- ([shift={(0.355cm, -0.55cm)}] Mux.east) |- (Mux.east); % Logic -> Mux
		\draw[conn] ([xshift=-0.1cm] Mux.north east) |- (FPGA_tx0); % Mux -> tx0

		% Intersections
		\fill[name intersections={of=FPGA_rx0--FPGA_tx1 and FPGA_rx0--Receiver2_din, total=\t}] (intersection-\t) circle (0.4mm);
		\fill[name intersections={of=FPGA_rx1--Mux and FPGA_rx1--Receiver_din, total=\t}] (intersection-\t) circle (0.4mm);
\end{tikzpicture}
        \caption{Timing attack.}
        \label{fig:as5-schematic}
        \vspace{1em}\hrule
    \end{center}
\end{figure}

    
    Installing VA testbacnhes.
    
    
    \section*{Assignment 2}

In the second Assignment we should connect a USB Input to an output-Pin of the board and an input Pin on the Board, to the Output via USB. Furthermore we had to activate another Reset-output-Pin when we read a specific sign via USB. The Validation of the correct Implementation should be done with the oscilloscope.

Our finished throughput contained a single UART-reciver. Practically the throughput was realized through wireing the USB-input to the output-pin and wireing the input-pin to the USB output. The only purpose of the UART-reciver was to check the USB-input for a specific ASCII-sign which was allocated to activate the reset function. This function would be needet in future assignments, see figure \ref{fig:as4-schematic-1} (p.~\pageref{fig:as4-schematic-1}) If the UART-reciver detected the demanded ASCII-singn the reset-pin would be set from \verb+0+ to \verb+1+. Even though this assignment looks like an easy task, the challenge was the assigning of in an output on the Papilo board. Furthermore the introduction to the oscilloscope and the first time use of this gadget was a challenge in itself.




%\emph{Bearbeitung und Lösung: } Der fertige Throughput arbeitete mit einem einzelnen UART-Reciver. Prinzipiell wurde der Throughput durch das verbinden der ein und ausgänge ralisiert. Der UART-Rciver diente zum erkennen eines besondern Zeichens, um dann wenn diese Zeichen erkannt wurde über einen gesonderten Ausgang ein Reset Signal zu schicken. Diese Funktion würde für Zukünftige Projekte wichtig sein.
%Die Schwierigkeit an Diesem Assignement, war das erstmalige auseinandersetzen mit dem zuweisen von Ein und Ausgängen auf dem Papilo Board. Außerdem wurde die Funktionalität des Throughputs mit Hilfe eines Osziluskopen überprüft. mit dessen Funktionsweise sich auseinander gesetzt werden musste.

    
    throughput
    
    \section*{Assignment 3}
The traffic lights hack.

\begin{figure}
    \begin{center}
        \begin{tikzpicture}[node distance=5cm]
%	\draw[gray,very thin] (0,0) grid (11,7); % Grid for easy orientation
	\tikzstyle{component} = [
		rectangle, draw=black, minimum width=3cm, minimum height=3cm, above right
	]
	\tikzstyle{component_caption} = [
		below right
	]

	\node (n1) [component] {}
		node [below right] at (n1.south west) {\textsf{\textbf{Computer}}}
		coordinate [label={below left:\texttt{tx}}] (computer_tx) at (3,2) % oben rechts
		coordinate [label={above left:\texttt{rx}}] (computer_rx) at (3,1) % unten rechts
	;

	\node (n2) [component, right of=n1] {}
		node [below right] at (n2.south west) {\textsf{\textbf{FPGA}}}
		node (r1) [below=.3cm ] at (n2.133){rx} % oben links : 180 - 45
		node (t2) [below=.3cm] at (n2.45){tx} % oben rechts
		node [above left] at (n2.-45){rx} % unten rechts
		node [above right] at (n2.225){tx} % unten links : 180 + 45
	;

	\draw[->] (0,2) -- (computer_tx);
	\draw[->] (0,1) -- (computer_rx);
	\node (n3) [component, right of=n2] {}
			node [below right] at (n3.south west) {\textsf{\textbf{Lock}}}
	;
\end{tikzpicture}
        \caption{Schematic of Ampel Hack}
    \end{center}
\end{figure}


Ampel
3 FPGA's 2 with ampel the other as Translator 
at 1st only the 2 ampels, watching with oszyluscope how they communicate.
at 2nd communicating with on ampel though the fpga transmitting asci keys to imitate the 2nd ampel
3rd Man in the middle 

3modes both green
both red
normal working

    \section*{Assignment 4}
The task was to bruteforce a 4 digit pin.

\emph{Aufgabe: }We had an FPGA configured by the tutors. We should enter it, without any Knowledge what it does.

\begin{figure}
    \begin{center}
        \usetikzlibrary{arrows.meta}
\begin{tikzpicture}
%	\draw[gray,very thin] (0,0) grid (11,7); % Grid for easy orientation
	\tikzstyle{comp} = [
		rectangle, draw=black
	]
	\tikzstyle{component} = [
		comp, minimum width=5.5cm, minimum height=4.5cm
	]
\tikzstyle{component_small} = [
		comp, minimum width=2cm, minimum height=2cm
	]
	\tikzstyle{caption} = [
		below right
	]
	\tikzstyle{conn} = [
		-{Latex[length=2mm]}
	]
	
	% FPGA
	\node (FPGA) [component] at (0,0) {}
		% Caption
		node [caption] at (FPGA.south west) { \textsf{\textbf{FPGA}} }
		% Interfaces linke Seite
		coordinate [yshift=2.5cm+0.666cm, label={ above right : \footnotesize{$\texttt{rx}_0$} }] (FPGA_rx0) at (FPGA.south west) % unten links
		coordinate [yshift=2.5cm+1.333cm, label={ above right : \footnotesize{$\texttt{tx}_0$} }] (FPGA_tx0) at (FPGA.south west) % oben links
		% Interfaces rechte Seite
		coordinate [yshift=2.5cm+0.666cm, label={ above left : \footnotesize{$\texttt{tx}_1$} }] (FPGA_tx1) at (FPGA.south east) % unten links
		coordinate [yshift=2.5cm+1.333cm, label={ above left : \footnotesize{$\texttt{rx}_1$} }] (FPGA_rx1) at (FPGA.south east) % oben links
	;

% Computer
	\node (Computer) [component_small, below left, xshift=-1cm] at (FPGA.north west) {}
		% Caption
		node [caption] at (Computer.south west) { \small{\textsf{\textbf{Computer}}} }
		% Interfaces
		coordinate [yshift=0.666cm, label={ left:\footnotesize{\texttt{tx}} }] (computer_tx) at (Computer.south east) % unten rechts
		coordinate [yshift=1.333cm, label={ left:\footnotesize{\texttt{rx}} }] (computer_rx) at (Computer.south east) % oben rechts
	;

	% Lock
	\node (Lock) [component_small, below right, xshift=1cm] at (FPGA.north east) {}
		% Caption
		node [caption] at (Lock.south west) { \textsf{\textbf{Lock}} }
		% Interfaces
		coordinate [yshift=0.666cm, label={ right:\footnotesize{\texttt{rx}} }] (lock_rx) at (Lock.south west) % unten links
		coordinate [yshift=1.333cm, label={ right:\footnotesize{\texttt{tx}} }] (lock_tx) at (Lock.south west) % oben links
	;

	% Receiver
	\node (Receiver) at (FPGA.south west) [component_small, above right, shift={(0.5, 0.75)}] {}
		% Caption
		node [caption] at (Receiver.south west) { \textsf{\footnotesize{\textbf{UART Recv.}}} }
		% Interfaces linke Seite
		coordinate [yshift=1cm, label={ right : \scriptsize{\texttt{din}} }] (din) at (Receiver.south west) % oben links
		% Interfaces rechte Seite
		coordinate [yshift=1.333cm+0.15cm, label={ left : \scriptsize{\texttt{data\_out}} }] (data_out) at (Receiver.south east)
		coordinate [yshift=1.333cm-0.15cm, label={ left : \scriptsize{$[7:0]$} }] (data_out2) at (Receiver.south east)
		coordinate [yshift=0.666cm, label={ left : \scriptsize{\texttt{valid}} }] (valid) at (Receiver.south east)
		% Internal connection endpoints
		coordinate [shift={(0.5cm,1.333cm)}] (data_out_end) at (Receiver.south east)
	;

	% Computer -> FPGA
	\draw[conn] (computer_tx) -- (FPGA_rx0);
	\draw[conn] (FPGA_rx0) -- (FPGA_tx1);
	\draw[conn] (FPGA_tx1) -- (lock_rx);
	% FPGA -> Computer
	\draw[conn]  (lock_tx) -- (FPGA_rx1) ;
	\draw[conn]  (FPGA_rx1) -- (FPGA_tx0);
	\draw[conn]  (FPGA_tx0) -- (computer_rx);
	% FPGA internal
	\draw[conn] (FPGA_rx0) +(0.2,0)|- (din);
	\draw[conn, thick] (data_out) +(0, -0.15) -- (data_out_end);

\end{tikzpicture}
        \caption{In this setup, all possible PINs are computed on the computer and transmitted to the device; the FPGA just relays the communication. But it also listens to the computers output bitstream to reset the mircrocontroller upon receiving a special command.}
    \end{center}
\end{figure}

\emph{Bearbeitung und Lösung: }Wenn die Von den Tutoren Programmierte FPGA (FPGA-PIN) über den Throughput an den Computer anschloss und die resettaste des FPGA-PIN betätigte, Würde man auf dem Cmputerbildschirm zur eingabe eines PINs aufgefordert. Nach der eingabe von Vier Ziffern zeigte der Computer an das der Pin falsch sei \emph{"Incorrect Pin"}. erst nache inigen Sekunden war es wieder möglich erneut einen Pin einzugeben. 
Der Lösungsansatz war trozdem eine Implementierung eines Brutforce algotythmus. Um die Wartezeit nach einer Falscheingabe zu umgehen, wurde jedoch der Reset Pin des FPGA-Pin angeprochen um die FPGA zu resetten und so sofort einen neuen Pin eingeben zu können. 
Um den Brutforce algorythmus auf dem FPGA zu realisieren, Bauten wir eine State Mashine mit 4 Zuständen, einem UART-Transmitter und einem UART-Reciver.

\begin{figure}
    \begin{center}
        \usetikzlibrary{arrows.meta}
\begin{tikzpicture}
	\tikzstyle{comp} = [
		rectangle, draw=black, thick
	]
	\tikzstyle{component} = [
		comp, minimum width=6cm, minimum height=5cm
	]
\tikzstyle{component_small} = [
		comp, minimum width=2cm, minimum height=2cm
	]
	\tikzstyle{caption} = [
		below right
	]
	\tikzstyle{conn} = [
		-{Latex[length=2mm]}
	]
	
	% FPGA
	\node (FPGA) [component] at (0,0) {}
		% Caption
		node [caption] at (FPGA.south west) { \small{\textsf{\textbf{FPGA}}} }
		% In/-outputs links
		coordinate [yshift=3cm+0.666cm, label={ above right : \footnotesize{} }] (FPGA_rx0)                           at (FPGA.south west) % unten
		coordinate [yshift=3cm+1.333cm, label={ above right : \footnotesize{$\texttt{tx}_0$} }] (FPGA_tx0) at (FPGA.south west) % oben
		% In/outputs  rechts
		coordinate [yshift=3cm,                    label={ above left : \footnotesize{$\texttt{rx}_1$} }]      (FPGA_rx1)        at (FPGA.south east)  % unten
		coordinate [yshift=3cm+0.666cm, label={ above left : \footnotesize{$\texttt{tx}_1$} }]      (FPGA_tx1)        at (FPGA.south east) % mitte
		coordinate [yshift=3cm+1.333cm, label={ above left : \footnotesize{$\texttt{restart}$} }] (FPGA_restart) at (FPGA.south east) % oben
	;

	% Logic
	\node (Logic) at (FPGA.north) [comp, minimum height=1cm, minimum width=2cm, below, shift={(-0.45, -0.25)}] {}
		node [caption] at (Logic.south west) { \textsf{\footnotesize{\textbf{Logic}}} }
		% Output rechts
		coordinate [yshift=0.583cm] (Logic_out0) at (Logic.south east)
		% Input unten
		coordinate [xshift=-0.275cm] (Logic_in0) at (Logic.south east)
		coordinate [xshift=-0.475cm] (Logic_in1) at (Logic.south east)
		coordinate [xshift=-0.75cm] (Logic_out1) at (Logic.south east)
		coordinate [xshift=-0.6cm] (Logic_out2) at (Logic.south east)
	;

	% Receiver
	\node (Receiver) at (FPGA.south east) [component_small, above left, shift={(-0.5, 0.75)}] {}
		% Caption
		node [caption] at (Receiver.south west) { \textsf{\footnotesize{\textbf{UART Recv.}}} }
		% Input rechts
		coordinate [yshift=1cm, label={ left : \scriptsize{\texttt{din}} }] (Receiver_din) at (Receiver.south east)
		% Outpus rechts
		coordinate [yshift=0.666cm,                 label={ right : \scriptsize{\texttt{valid}} }]           (Receiver_valid)           at (Receiver.south west) % unten
		coordinate [yshift=1.333cm+0.15cm, label={ right : \scriptsize{\texttt{data\_out}} }] (Receiver_data_out)    at (Receiver.south west) % oben
		coordinate [yshift=1.333cm-0.15cm,  label={ right : \scriptsize{$[7:0]$} }]                     (Receiver_data_out2) at (Receiver.south west) % mitte
	;

	% Transmitter
	\node (Transmitter) at (FPGA.south west) [component_small, above right, shift={(0.5, 0.75)}] {}
		node [caption] at (Transmitter.south west) { \textsf{\footnotesize{\textbf{UART Trans.}}} }
		% Output links
		coordinate [yshift=1cm, label={ right: \scriptsize{\textsf{\texttt{dout}}} }] (Transmitter_dout) at (Transmitter.south west) % unten
		% Inputs links
		coordinate [yshift=0.666cm,                 label={ left : \scriptsize{\texttt{enable}} }]           (Transmitter_enable)           at (Transmitter.south east) % unten
		coordinate [yshift=1.333cm+0.15cm, label={ left : \scriptsize{\texttt{data\_in}} }] (Transmitter_data_in)    at (Transmitter.south east) % oben
		coordinate [yshift=1.333cm-0.15cm,  label={ left : \scriptsize{$[7:0]$} }]                     (Transmitter_data_in2) at (Transmitter.south east) % mitte
	;

	% Computer
	\node (Computer) [component_small, below left, xshift=-1cm] at (FPGA.north west) {}
		% Caption
		node [caption] at (Computer.south west) { \small{\textsf{\textbf{Computer}}} }
		% In/outputs rechts
		coordinate [yshift=0.666cm, label={ left:\footnotesize{\texttt{tx}} }] (Computer_tx) at (Computer.south east) % unten
		coordinate [yshift=1.333cm, label={ left:\footnotesize{\texttt{rx}} }] (Computer_rx) at (Computer.south east) % oben
	;

	% Lock
	\node (Lock) [component_small, below right, xshift=1cm] at (FPGA.north east) {}
		% Caption
		node [caption] at (Lock.south west) { \small{\textsf{\textbf{Lock}}} }
		% In/outputs rechts
		coordinate [yshift=0.333cm, label={ right:\footnotesize{\texttt{tx}} }]   (Lock_tx)   at (Lock.south west) % unten
		coordinate [yshift=0.999cm, label={ right:\footnotesize{\texttt{rx}} }]   (Lock_rx)   at (Lock.south west) % mitte
		coordinate [yshift=1.666cm, label={ right:\footnotesize{\texttt{rst}} }] (Lock_rst) at (Lock.south west) % oben

	;

	% Computer -> FPGA
	\draw[conn] (Computer_tx) -- (FPGA_rx0);
	\draw[conn] (FPGA_tx1) -- (Lock_rx);
	\draw[conn] (FPGA_restart) -- (Lock_rst);
	
	% FPGA -> Computer
	\draw[conn]  (Lock_tx) -- (FPGA_rx1) ;
	\draw[conn]  (FPGA_tx0) -- (Computer_rx);
	
	% FPGA internal
	\coordinate[xshift=-0.2cm] (h1) at (Transmitter_dout);
	\coordinate[xshift=0.3cm] (h2) at (FPGA_tx0);
 	\draw[conn]  (Transmitter_dout) -- (h1) -- (h2) -- (FPGA_tx0);
	\coordinate[shift=({4.8cm, -1.166cm})] (ha) at  (FPGA_tx0) ;
	\coordinate[yshift=0.5cm] (hb) at  (ha) ;
	\draw[conn] (FPGA_tx0) +(0.3, -1.166) -- (ha) -- (hb) -- (FPGA_tx1);
 	\draw[fill=black] (FPGA_tx0) +(0.3,-1.166) circle (0.4mm);

	\coordinate[xshift=-0.2cm] (h3) at (FPGA_rx1);
	\coordinate[xshift=0.3cm] (h4) at (Receiver_din);
	\draw[conn] (FPGA_rx1) -- (h3) -- (h4) -- (Receiver_din);

	\draw[conn] (Logic_out0) -- (FPGA_restart);

	\coordinate[xshift=-0.2cm] (h5) at (Receiver_data_out);
	\draw[conn, very thick] (Receiver_data_out) -- (h5) -- (Logic_in0);

	\coordinate[xshift=-0.4cm] (h6) at (Receiver_valid);
	\draw[conn] (Receiver_valid) -- (h6) -- (Logic_in1);

	\coordinate[xshift=0.45cm] (h8) at (Transmitter_enable);
	\draw[conn] (Logic_out2) -- (h8) -- (Transmitter_enable);

	\coordinate[xshift=0.3cm] (h7) at (Transmitter_data_in);
	\draw[conn, very thick] (Logic_out1) -- (h7) -- (Transmitter_data_in);
\end{tikzpicture}
        \caption{Now the FPGA calculates the 4 digit numbers, sends them to the device and decodes the answer bitstream in order to reset the microcontroller. The computer is only needed as output device in this case.}
    \end{center}
\end{figure}


\begin{itemize}


\item[START] Ist der erste Zustand der STate mashine, in dem die 4 stellen des Pins sowie einem 5. Übertragszahl auf 0 gesetzt. diese 5 Ziffern werden als 8-Bit gehandhabt. Außerdem werden andere Register initialisiert.

\item[SEND] In diesem Zustand das erste Byte des 4 Stelligen Pins an den UART-Tranmitter übergeben, welcher es direkt an die FPGA-Pin weiterleitet. nun werden alle stellen des Pins um eine stelle verrückt, wobei die vorher erste stelle auf die Hilfvariable übertragen wird. Nun wird der Vorgang wiederholt und die neue erste Stelle des Pins via Transmitter übermittelt. Dieser vorgang findet statt, bis alle zeichen wieder an ihrem orginalen ort sind. 

\item[INC] Hier wird Die nidrigste Stelle des Pins um 1 erhöht. ist diese Stelle bereits 9 muss der übertrag berücksichtigt werden. Außerdem wird der Aktuelle Pin an den Computer Übergeben, um den Vortgang verfolgen zu können.

\item[RECIVE] in diesem Zustand wird auf die Antwort des FPGA-Pin gewartet. Beginnt die Antwort mit einem "I" wie bei Incorrect Pin, wird die Resetfunktion des FPGA-Pin betätigt und die Statmashine begibt sich wieder in den SEND zustand und setzt die Bruteforce Attacke fort.

\end{itemize}




Pin 
resetting fpga, tou cut the time when fpga was responding wrong PW
the rest was implemented via python, and tee Fpga worked as Translator

    \section*{Assignment 5}
% Password.
% Fpga was counting time ow long the responsetime was for responding, waiting for the 1st didgit of the answere. middelng the time between als answeres to get the longes one right. using thatone as correct didgit of the PW.
%
% building the counter 4 the clockcycles was the hard part strings where to long
% fpga giving back the time  we had to use 2 bytes
% 
% python did the rest.

Like in assignment 4, we get a microcontroller (we know, that its sowftware is written using the c programming language) and analyse it the same way as before. 
This time it directly prompts \textit{``Please enter password:''}, answering \textit{``Password incorrect''} to our guesses and the goal is set. As a possible attack vector comes a timing vulnerability to mind, if the password comparison is implemented using \texttt{strcmp}. We continue setting up the FPGA like before -- able to reset the device on command and relaying all communication. We just need a little further testing to be sure, that we are indeed going to program a timing attack.

To measure the timing of the string comparison on the micro controller, we have to know when it is actually executed, or, in other words how long it takes the device to send \textit{``Please enter password:''} leading to it being in a state, where it excepts input. 
Again we use an oscilloscope hooked to the \texttt{tx} line of the device to obtain this timing offset. 

Remember, at this point the FPGA is already set up like in figure \ref{fig:as4-schematic-1} (p.~\pageref{fig:as4-schematic-1}).
We add another receiver (to whats comming from the device), and a transmitter (to the computer), tell the FPGA in a finite 3-state machine to start the timer on receiving `:', stop it on receiving `i', and transmit the measurement to the computer afterwards, as shown in figure \ref{fig:as5-schematic} (p.~\pageref{fig:as5-schematic}). Technically, that is achieved by counting the clock cycles between both events in a 2 byte integer; if the FPGA is relaying the micro controllers output, or sending the measured timing is determined by a multiplexer.

Now a Python script can launch the attack, bruteforcing every single sign at a time, again:

\begin{enumerate}
    \item Wait the timing offset;
    \item transmit the current password guess;
    \item receive 2 bytes timing score;
    \item add the sign with the highest timing score to the known password.
\end{enumerate}

We actually take several measurements for every password guess and take the mean in order to be less error-prone.


\begin{figure}[h!]
    \begin{center}
%        \hrule\vspace{1em}
        \usetikzlibrary{arrows.meta}
\usetikzlibrary{calc,intersections,through,backgrounds}
\begin{tikzpicture}
% 	\tikzset{
% 	  every node/.style={scale=1.1}
% 	}	
	\tikzset{comp/.style={
		rectangle, draw=black, thick
	}}	
	\tikzset{component/.style={
		comp, minimum width=6cm, minimum height=6cm, very thick
	}}
	\tikzset{component_small/.style={
		comp, minimum width=2cm, minimum height=2cm, thick
	}}
	\tikzset{component_tiny/.style={
		comp, inner sep=0.1cm, semithick, right
	}}
	\tikzset{caption/.style={
		below right
	}}
	\tikzset{conn/.style={
		-{Latex[length=2mm]}
	}}
	
	% FPGA
	\node (FPGA) [component] at (0,0) {}
		% Caption
		node [caption] at (FPGA.south west) { \small{\textsf{\textbf{FPGA}}} }
		
		% In/-outputs links
		coordinate [yshift=4cm+0.4pt+0.666cm, label={ above right : \footnotesize{$\texttt{rx}_0$} }] (FPGA_rx0) at (FPGA.south west) % unten
		coordinate [yshift=4cm+0.4pt+1.333cm, label={ above right : \footnotesize{$\texttt{tx}_0$} }] (FPGA_tx0) at (FPGA.south west) % oben

		% In/outputs  rechts
		coordinate [yshift=4cm+0.4pt,                    label={ above left : \footnotesize{$\texttt{rx}_1$} }]      (FPGA_rx1)       at (FPGA.south east)  % unten
		coordinate [yshift=4cm+0.4pt+0.666cm, label={ above left : \footnotesize{$\texttt{tx}_1$} }]      (FPGA_tx1)       at (FPGA.south east) % mitte
		coordinate [yshift=4cm+0.4pt+1.333cm, label={ above left : \footnotesize{$\texttt{restart}$} }] (FPGA_restart) at (FPGA.south east) % oben
		
		% Internals
		node (Mux)          [component_tiny, shift={(0.25cm,-1.8cm) }]  at (FPGA.north west)   { \scriptsize{\textsf{\texttt{Mux}}} }
	;

	% Logic
	\node (Logic) at (FPGA.north) [comp, minimum height=1cm, minimum width=1.5cm, below right, shift={(0.6cm, -0.75cm)}] {}
		node [above right] at (Logic.south west) { \textsf{\footnotesize{\textbf{Logic}}} }
	;

	% Receiver
	\node (Receiver) at (FPGA.south east) [component_small, above left, shift={(-0.5, 0.75)}] {}
		% Caption
		node [caption] at (Receiver.south west) { \textsf{\footnotesize{\textbf{UART Recv.}}} }

		% Input rechts
		coordinate [yshift=1cm, label={ left : \scriptsize{\texttt{din}} }] (Receiver_din) at (Receiver.south east)

		% Outpus links
		coordinate [yshift=0.666cm,                 label={ right : \scriptsize{\texttt{valid}} }]           (Receiver_valid)           at (Receiver.south west) % unten
		coordinate [yshift=1.333cm+0.15cm, label={ right : \scriptsize{\texttt{data\_out}} }] (Receiver_data_out)    at (Receiver.south west) % oben
		coordinate [yshift=1.333cm-0.15cm,  label={ right : \scriptsize{$[7:0]$} }]                     (Receiver_data_out2) at (Receiver.south west) % mitte
	;

	% Receiver2
	\node (Receiver2) at (FPGA.south west) [component_small, above right, shift={(0.5, 0.75)}] {}
		% Caption
		node [caption] at (Receiver2.south west) { \textsf{\footnotesize{\textbf{UART Recv.}}} }

		% Input rechts
		coordinate [yshift=1cm, label={ right : \scriptsize{\texttt{din}} }] (Receiver2_din) at (Receiver2.south west)

		% Outpus links
		coordinate [yshift=0.666cm,                 label={ left : \scriptsize{\texttt{valid}} }]           (Receiver2_valid)           at (Receiver2.south east) % unten
		coordinate [yshift=1.333cm+0.15cm, label={ left : \scriptsize{\texttt{data\_out}} }] (Receiver2_data_out)    at (Receiver2.south east) % oben
		coordinate [yshift=1.333cm-0.15cm,  label={ left : \scriptsize{$[7:0]$} }]                     (Receiver2_data_out2) at (Receiver2.south east) % mitte
	;

	% Transmitter
	\node (Transmitter) at (FPGA.north west) [component_small, below right, shift={(1.35, -0.25)}] {}
		node [caption] at (Transmitter.south west) { \textsf{\footnotesize{\textbf{UART Trans.}}} }

		% Output links
		coordinate [yshift=1cm, label={ right: \scriptsize{\textsf{\texttt{dout}}} }] (Transmitter_dout) at (Transmitter.south west) % unten

		% Inputs links
		coordinate [yshift=0.666cm,                 label={ left : \scriptsize{\texttt{enable}} }]    (Transmitter_enable)   at (Transmitter.south east) % unten
		coordinate [yshift=1.333cm-0.15cm,  label={ left : \scriptsize{$[7:0]$} }]                  (Transmitter_data_in2)at (Transmitter.south east) % mitte
		coordinate [yshift=1.333cm+0.15cm, label={ left : \scriptsize{\texttt{data\_in}} }] (Transmitter_data_in)  at (Transmitter.south east) % oben	
	;

	% Computer
	\node (Computer) [component_small, below left, xshift=-1cm] at (FPGA.north west) {}
		% Caption
		node [caption] at (Computer.south west) { \small{\textsf{\textbf{Computer}}} }

		% In/outputs rechts
		coordinate [yshift=0.666cm, label={ left:\footnotesize{\texttt{tx}} }] (Computer_tx) at (Computer.south east) % unten
		coordinate [yshift=1.333cm, label={ left:\footnotesize{\texttt{rx}} }] (Computer_rx) at (Computer.south east) % oben
	;

	% Lock
	\node (Lock) [component_small, below right, xshift=1cm] at (FPGA.north east) {}
		% Caption
		node [caption] at (Lock.south west) { \small{\textsf{\textbf{Lock}}} }

		% In/outputs rechts
		coordinate [yshift=0.333cm, label={ right:\footnotesize{\texttt{tx}} }]   (Lock_tx)   at (Lock.south west) % unten
		coordinate [yshift=0.999cm, label={ right:\footnotesize{\texttt{rx}} }]   (Lock_rx)   at (Lock.south west) % mitte
		coordinate [yshift=1.666cm, label={ right:\footnotesize{\texttt{rst}} }] (Lock_rst)  at (Lock.south west) % oben
	;

	% Computer <-> FPGA
	\draw[conn]  (FPGA_tx0) -- (Computer_rx);
	\draw[conn] (Computer_tx) -- (FPGA_rx0);

	% FPGA <-> Lock
	\draw[conn] (FPGA_restart) -- (Lock_rst);
	\draw[conn] (FPGA_tx1) -- (Lock_rx);
	\draw[conn] (Lock_tx) -- (FPGA_rx1) ;
	
	% FPGA internal
		\draw[conn] (Logic.east) -| ([shift={(0.15cm, 0.566cm)}] Logic.east) -- (FPGA_restart);
		\draw[conn, name path=FPGA_rx0--FPGA_tx1] (FPGA_rx0) -| ([shift={(0.2cm, -1.375cm)}] FPGA_rx0) -- ([shift={(5.3cm, -1.375cm)}] FPGA_rx0) |- (FPGA_tx1); % Pass through Computer -> Lock
	
		% Connections to/from Receiver
		\draw[conn, name path=FPGA_rx1--Receiver_din] (FPGA_rx1) -- ([xshift=-0.2cm] FPGA_rx1) |- (Receiver_din); % rx1 -> Logic
		\draw[conn, very thick] (Receiver_data_out) -|  ([shift={(-0.2cm,0.6cm)}] Receiver_data_out) -| ([xshift=0.3cm] Logic.south);  % data_out -> Logic
		\draw[conn] (Receiver_valid) -| ([shift={(-0.4cm, 1.5cm)}] Receiver_valid) -| ([xshift=0.1cm] Logic.south); % valid -> Logic
		
		% Connections to/from Receiver2
		\draw[conn, name path=FPGA_rx0--Receiver2_din] (FPGA_rx0) -- ([xshift=0.2cm] FPGA_rx0) |- (Receiver2_din); % rx0 -> din
		\draw[conn, very thick] (Receiver2_data_out) -| ([shift={(0.2cm,0.9cm)}] Receiver2_data_out) -| ([xshift=-0.3cm] Logic.south); % data_out -> Logic 
		\draw[conn] (Receiver2_valid) -| ([shift={(0.4cm, 1.6cm)}] Receiver2_valid) -| ([xshift=-0.1cm] Logic.south); % valid -> Logic

		% Connections to/from Transmitter
		\draw[conn, very thick] ([yshift=-0.192cm] Logic.north west) -- ([yshift=-0.15cm] Transmitter_data_in); % Logic -> data_in
		\draw[conn] ([yshift=0.166cm] Logic.south west) -- (Transmitter_enable); % Logic -> enable
		\draw[conn] (Transmitter_dout) -| ([shift={(-0.15cm, -1.3cm)}] Transmitter_dout) -| ([xshift=0.1cm] Mux.south); % Transmitter -> Mux

		% Connections to/from Mux
		\draw[conn, name path=FPGA_rx1--Mux]  (FPGA_rx1) -- ([xshift=-0.5cm] FPGA_rx1) |-  ([shift={(-0.2cm, -1.55cm)}]  Transmitter_dout) -| ([xshift=-0.1cm] Mux.south); % rx1 -> Mux
		\draw[conn] ([xshift=-0.5cm] Logic.south) |- ([shift={(0.355cm, -0.55cm)}] Mux.east) |- (Mux.east); % Logic -> Mux
		\draw[conn] ([xshift=-0.1cm] Mux.north east) |- (FPGA_tx0); % Mux -> tx0

		% Intersections
		\fill[name intersections={of=FPGA_rx0--FPGA_tx1 and FPGA_rx0--Receiver2_din, total=\t}] (intersection-\t) circle (0.4mm);
		\fill[name intersections={of=FPGA_rx1--Mux and FPGA_rx1--Receiver_din, total=\t}] (intersection-\t) circle (0.4mm);
\end{tikzpicture}
        \caption{Timing attack.}
        \label{fig:as5-schematic}
        \vspace{1em}\hrule
    \end{center}
\end{figure}


Password. Fpga was counting time ow long the responsetime was for responding, waiting for the 1st didgit of the answere. middelng the time between als answeres to get the longes one right. using thatone as correct didgit of the PW.

building the counter 4 the clockcycles was the hard part strings where to long
fpga giving back the time  we had to use 2 bytes
 
python did the rest.

    
\end{document}
