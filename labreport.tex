\documentclass[a4paper, twoside, ngerman, 12pt]{scrartcl}
\usepackage{amssymb}
\usepackage{ifxetex}
\ifxetex
    % XeLaTeX
    \usepackage{fontspec,xunicode,xltxtra}
    \usepackage{polyglossia}
    \setmainlanguage[spelling=new,babelshorthands=true]{german}
    \setsansfont{Gillius ADF}
    \usepackage{ebgaramond}
    \usepackage[]{unicode-math}
    %    \setmainfont{XITS}
    %    \setmathfont{XITS Math}
\else
    % default: pdfLaTeX
    \usepackage[utf8]{inputenc}
    \usepackage[ngermanb]{babel}
    \usepackage[T1]{fontenc}
    \usepackage[babel=true]{microtype}
    \usepackage{lmodern}
\fi
\usepackage{amsmath}

\usepackage{eurosym}
\usepackage{parskip}
\usepackage{svg}
\usepackage{lscape}
\usepackage[inner=2.3cm,outer=4.6cm,top=3.3cm,bottom=6.6cm]{geometry}

\usepackage[nottoc]{tocbibind}
\usepackage[fixlanguage]{babelbib}
%\selectbiblanguage{german}

\usepackage{graphicx}
\usepackage{caption}
\usepackage{xcolor}
    \definecolor{lblue}{cmyk}{0.92, 0.7, 0, 0}            % #2020c0
\usepackage[hyphens]{url}
\usepackage[bookmarks]{hyperref}
    \hypersetup{
        colorlinks=true,  % false: boxed links; true: colored links
        linkcolor=lblue,   % color of internal links
        citecolor=lblue,   % color of links to bibliography
        filecolor=lblue,   % color of file links
        urlcolor=lblue     % color of external links
    }

\usepackage{fancyhdr}
\setlength{\headheight}{15pt}
\pagestyle{fancy}
\fancyhf{}
\lhead[\nouppercase\leftmark]{}
\rhead[]{\nouppercase\rightmark}
\lfoot[\thepage]{}
\rfoot[]{\thepage}

%\newfontfamily\sectionfont[Color=MSLightBlue]{Times New Roman}

\linespread{1.2} % Zeilenhöhe wird auf Faktor 1.25 gesetzt
\setlength{\parindent}{0cm}
\graphicspath{{img/}}
%\renewcommand*{\figureformat}{}
% Die Boxen zum zählen der Verdrängungsindikatoren als Matheoperatoren, damit das spacing einheitlich ist
\DeclareMathOperator{\bsq}{\blacksquare}
\DeclareMathOperator{\wsq}{\square}
\usepackage{pifont}
%\renewcommand{\labelitemi}{\ding{112}} % \blacktriangleright

\title{Hardware Security}
\subtitle{Labreport}


\date{}
\begin{document}
    \pagenumbering{roman}
    \begin{titlepage}
    \newgeometry{inner=3.45cm,outer=3.45cm,top=3.3cm,bottom=4.6cm}
        \begin{figure*}
            \begin{flushright}
%                \includegraphics[width=0.4\textwidth]{img/Logokurz.pdf}
            \end{flushright}
        \end{figure*}
        \vspace*{-1cm}
        \hrule
        \vspace{.2cm}
        \begin{center}
            \vfill
            \textsf{\textbf{\Huge{Hardware Security}}}\\[1.5em]
            \textsf{\LARGE{Labreport}}
            \vfill
        \end{center}
        \vfill
        \hrule
        \vspace{.2cm}
        \large{Berlin, October 2016}
    \end{titlepage}
    
    \pagenumbering{arabic}
    \setcounter{page}{1}
    
    \section*{Assignment 1}
    Hier steht ein Testtext
%    \input{tex/1-darstellung.tex}
    
    \section*{Assignment 2}
%    \input{tex/2-gefluechtete.tex}
    
    \section*{Assignment 3}
%    \input{tex/3-aktive.tex}
    
    \section*{Assignment 4}
%    \input{tex/4-sonstiges.tex}
    
    \section*{Assignment 5}
\end{document}
