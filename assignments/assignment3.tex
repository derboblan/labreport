\section*{Assignment 3}
Two identically programmed micro controllers equiped with three LEDs and communicating via UART are handed to us. [What do we see observing them?] -- they resemble two simple traffic lights. In a first step we are asked to reverse engineer the underlying protocol. To do so, we hook an oscilloscope to the \texttt{rx} and \texttt{tx} lines of one device and observe its behaviour during and after a reset. We see the red light active and one single ASCII byte $47_{16}$, `G', being transmitted. [Conjecture how protocol works: It seems as if ...]. We test our hypothesis by connecting one micro controller directly to the computer: Whenever we send the letter `G' it instantly illuminates the green light and after some time turns to yellow then red staying like this if we do not interact further.

Now we are ready for the real assignment: launch a man-in-the-middle attack against the traffic lights protocol, implementing the three following modes of operation.
\begin{itemize}
    \item[] \textsf{Throughput}\footnote{Actually pass through}: the FPGA passively relays the communication between the lights like a cable (the default setting);
    \item[] \textsf{Green}: denial of service by setting both lights to \textcolor{green}{green};
    \item[] \textsf{Red}: denial of service by setting both lights to \textcolor{red}{red}. Leads to starvation because `G' is only sent once when activating the red light.
\end{itemize}

As a side requirement we want to be able to cycle through the modes seamlessly, i.~e., no matter which mode $M$ the FPGA is in, we can always choose the next operation $M'$ freely.

See figure \ref{fig:as3-schematic} (p.~\pageref{fig:as3-schematic}).

\begin{figure}[tb]
    \begin{center}
        \begin{tikzpicture}[node distance=5cm]
%	\draw[gray,very thin] (0,0) grid (11,7); % Grid for easy orientation
	\tikzstyle{component} = [
		rectangle, draw=black, minimum width=3cm, minimum height=3cm, above right
	]
	\tikzstyle{component_caption} = [
		below right
	]

	\node (n1) [component] {}
		node [below right] at (n1.south west) {\textsf{\textbf{Computer}}}
		coordinate [label={below left:\texttt{tx}}] (computer_tx) at (3,2) % oben rechts
		coordinate [label={above left:\texttt{rx}}] (computer_rx) at (3,1) % unten rechts
	;

	\node (n2) [component, right of=n1] {}
		node [below right] at (n2.south west) {\textsf{\textbf{FPGA}}}
		node (r1) [below=.3cm ] at (n2.133){rx} % oben links : 180 - 45
		node (t2) [below=.3cm] at (n2.45){tx} % oben rechts
		node [above left] at (n2.-45){rx} % unten rechts
		node [above right] at (n2.225){tx} % unten links : 180 + 45
	;

	\draw[->] (0,2) -- (computer_tx);
	\draw[->] (0,1) -- (computer_rx);
	\node (n3) [component, right of=n2] {}
			node [below right] at (n3.south west) {\textsf{\textbf{Lock}}}
	;
\end{tikzpicture}
        \caption{FPGA in the middle with three modes of operation \textsf{Throughput} (\textcolor{blue}{blue}), \textsf{Green} (\textcolor{green}{green}) and \textsf{Red} (\textcolor{red}{red}) controlled via the computer. The transmitter is always enabled and has a fixed input byte producing a cyclic bitstream \texttt{dout}. Outputs and hence operation mode are selected by multiplexers.}
        \label{fig:as3-schematic}
        \vspace{1em}\hrule
    \end{center}
\end{figure}
