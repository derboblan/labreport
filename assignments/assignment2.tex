\section*{Assignment 2}
Throughput and Work with the oszilluscope.

\emph{Aufgabe: } We should connect an USB Input to an output-Pin of the board and an input Pin on the Bouard, to the Output via USB. thurthermore we had to activate another Reset-output-Pin when we read a Specifc sign via USB. The Validation of the correct Implementation should be done with the osziluscope.


\emph{Bearbeitung und Lösung: } Der fertige Throughput arbeitete mit einem einzelnen UART-Reciver. Prinzipiell wurde der Throughput durch das verbinden der ein und ausgänge ralisiert. Der UART-Rciver diente zum erkennen eines besondern Zeichens, um dann wenn diese Zeichen erkannt wurde über einen gesonderten Ausgang ein Reset Signal zu schicken. Diese Funktion würde für Zukünftige Projekte wichtig sein.
Die Schwierigkeit an Diesem Assignement, war das erstmalige auseinandersetzen mit dem zuweisen von Ein und Ausgängen auf dem Papilo Board. Außerdem wurde die Funktionalität des Throughputs mit Hilfe eines Osziluskopen überprüft. mit dessen Funktionsweise sich auseinander gesetzt werden musste.
